\subsection{Big Number}
\begin{lstlisting}
struct bigNum{
	static const int L=1000;
	int it[L+10];
	bigNum(){
		fill(it, 0), it[0]=1;
	}
	bigNum(int n){
		fill(it, 0);
		while (n){
			it[++it[0]]=n%10;
			n/=10;
		}
		if (!it[0]) it[0]=1;
	}
	bigNum operator +(const bigNum & b)const{
		bigNum ret;
		ret.it[0]=max(it[0], b.it[0])+1;
		repf(i, 1, ret.it[0]){
			ret.it[i]+=it[i]+b.it[i];
			ret.it[i+1]+=ret.it[i]/10;
			ret.it[i]%=10;
		}
		while (ret.it[0]>1 && ret.it[ret.it[0]]==0) ret.it[0]--;
		return ret;
	}
	bigNum operator -(const bigNum & b)const{
		bigNum ret;
		ret.it[0]=it[0];
		repf(i, 1, ret.it[0]){
			ret.it[i]+=it[i]-b.it[i];
			if (ret.it[i]<0)
				ret.it[i]+=10, ret.it[i+1]--;
		}
		while (ret.it[0]>1 && ret.it[ret.it[0]]==0) ret.it[0]--;
		return ret;
	}
	bigNum operator *(const bigNum & b)const{
		bigNum ret;
		ret.it[0]=it[0]+b.it[0];
		repf(i, 1, it[0]) repf(j, 1, b.it[0])
			ret.it[i+j-1]+=it[i]*b.it[j];
		repf(i, 1, ret.it[0]) 
			ret.it[i+1]+=ret.it[i]/10, ret.it[i]%=10;			
		while (ret.it[0]>1 && ret.it[ret.it[0]]==0) ret.it[0]--;
		return ret;
	}
	void out(){
		repd(i, it[0], 1) printf("%d", it[i]);
		putchar('\n');
	}
};
\end{lstlisting}
