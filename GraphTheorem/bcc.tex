\subsection{Biconnected Component}
\begin{lstlisting}
vi a[N+10], bcc[N+10];
int pre[N+10], bccno[N+10], low[N+10];
bool iscut[N+10];
int dfs_clock, bcc_cnt;
stack<E> s;

void dfs(int i, int fa){
    pre[i]=low[i]=++dfs_clock;
    int child=0;
    rep(k, sz(a[i])){
        int j=a[i][k];
        if (!pre[j]){
            s.push(E(i,j)), child++;
            dfs(j, i), checkmin(low[i], low[j]);
            if (low[j]>=pre[i]){
                iscut[i]=true, bcc_cnt++, bcc[bcc_cnt].clear();
                for(;;){
                    E e=s.top(); s.pop();
                    if (bccno[e.i]!=bcc_cnt) bcc[bcc_cnt].pb(e.i);
                    if (bccno[e.j]!=bcc_cnt) bcc[bcc_cnt].pb(e.j);
                    bccno[e.i]=bccno[e.j]=bcc_cnt;
                    if (e.i==i && e.j==j) break;
                }
            }
        }
        else if (j!=fa && pre[j]<pre[i])
            s.push(E(i,j)), checkmin(low[i], pre[j]);
    }
    if (fa<0 && child==1) iscut[i]=0;
}


void find_bcc(int n){
    dfs_clock=bcc_cnt=0;
    clr(pre, 0), clr(iscut, 0), clr(bccno, 0);
	repf(i, 1, n) if (!pre[i])
    	dfs(i, -1);
}

\end{lstlisting}
