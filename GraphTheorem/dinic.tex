\subsection{Dinic}
\begin{lstlisting}
struct e_t{ 
	int to, cap, rev;
	e_t(int to, int cap, int rev):to(to), cap(cap), rev(rev){}
};

template<int SZ>
class Dinic{
public:
	vector<e_t> a[SZ+10];
	int lev[SZ+10], done[SZ+10];	
	int s, t;

	bool levelize(){
		queue<int> q; fill(lev, -1);
		q.push(s), lev[s]=0;
		while (!q.empty()){
			int i=q.front(); q.pop();
			rep(k, sz(a[i])){
				e_t e=a[i][k];
				if (!e.cap || lev[e.to]!=-1) continue;
				lev[e.to] = lev[i] + 1;
				q.push(e.to);
			}
		}
		return lev[t]!=-1;
	}

	int augment(int v, int f){
		if (v==t || !f) return f;
		for (; done[v] <sz(a[v]); ++done[v]){
			e_t &e = a[v][done[v]];
			if (lev[e.to] < lev[v] || !e.cap) continue;
			int t = augment(e.to, min(f, e.cap));
			if (t){
				e.cap -= t;
				a[e.to][e.rev].cap += t;
				return t;
			}
		}
		return 0;
	}

	void clear(){
		rep(i, SZ) a[i].clear();
	}

	void add(int i, int j, int c){
		a[i].pb(e_t(j, c, sz(a[j])));
		a[j].pb(e_t(i, 0, sz(a[i])-1));
	}

	int maxFlow(){
		int tot=0, tmp;
		while (levelize()){
			fill(done, 0);
			while (tmp = augment(s, INF))
				tot += tmp;
		}
		return tot;
	}
};
\end{lstlisting}
